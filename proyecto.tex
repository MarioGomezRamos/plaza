\documentclass[a4paper,12pt,twoside]{article}
\usepackage[latin1,utf8]{inputenc}
\usepackage[spanish]{babel}
\usepackage[T1]{fontenc}
\usepackage[usenames]{color}
\usepackage{amsmath}
\usepackage{amssymb}
\usepackage{setspace}
\usepackage{booktabs}
\usepackage{pifont}
\usepackage{hyperref}
\usepackage[spanish]{babelbib}
\usepackage[nottoc,numbib]{tocbibind}
\usepackage{enumitem}
\usepackage{eurosym}
\usepackage{epsfig}
\usepackage{textcomp}
\usepackage[a4paper,bindingoffset=0.cm,%
            left=2.5cm,right=2.5cm,top=3.5cm,bottom=2.5cm,]{geometry}
%\documentstyle[12pt]{article}
%\setlength{\textheight}{27.2cm}
%\setlength{\textwidth}{18.5cm}
%\setlength{\oddsidemargin}{2.5cm}
%\setlength{\evensidemargin}{2.5cm}
%\setlength{\topmargin}{2.5cm}

%\setlength{\parindent}{0pt}
%\addtolength{\topmargin}{-1.5cm}
%\addtolength{\oddsidemargin}{-1.8cm}
%\addtolength{\evensidemargin}{-1.cm}

%%%%%%% espacio entre lineas %%%%%%%%
%\renewcommand{\baselinestretch}{1.25}
%%%%%%%%punto decimal%%%%%%%%%%%%%%%%


%\def\ket#1{|#1\rangle}
%\def\bra#1{\langle#1|}
%\def\scal#1#2{\langle#1|#2\rangle}
%\def\matr#1#2#3{\langle#1|#2|#3\rangle}
%\def\bino#1#2{\left(\begin{array}{c}#1\\#2\end{array}\right)}
%\def\ave#1{\langle #1\rangle}
%\def\dis#1{\langle\langle #1\rangle\rangle}
%\def\uvo#1{\lq #1\rq\ }
%\def\uuvo#1{\lq\lq #1\rq\rq\ }
%\def\ave#1{\langle#1\rangle}
%\newcommand{\field}[1]{\mathbb{#1}}
%\newcommand{\ra}{\rangle}
%\newcommand{\la}{\langle}
%\newcommand{\rp}{\right)}
%\newcommand{\lp}{\left(}
%\newcommand{\rc}{\right]}
%\newcommand{\lc}{\left[}
%\newcommand{\arctanh}{\mbox{arctanh\hspace{0.05cm}}}
%\newcommand{\obeta}{\overline{\beta}}
%\newcommand{\onu}{\overline{\nu}}

%\usepackage{titlesec}


%\titleformat{\chapter}[hang]{\bf\huge}{\thechapter}{2pc}{}

\usepackage{fancyhdr}
 
\pagestyle{fancy}
\fancyhf{}
\fancyhead[RE,LO]{\underline{\nouppercase{\rightmark}}}
\fancyhead[LE,RO]{\thepage}
\renewcommand{\headrulewidth}{0pt}
%\rfoot{Page }
 
%\onehalfspacing
\spacing{1.5}

\addto\captionsspanish{\renewcommand{\tablename}{Tabla}}

\begin{document}

%\renewcommand{\refname}{Bibliografía}

\title{\textbf{Proyecto: Planteamientos docentes e investigadores}}
\author{Mario G\'omez Ramos}
\maketitle
%\begin{figure}
%\centering
%\includegraphics[height=3cm]{us.pdf}
%\end{figure}

\thispagestyle{empty}
\newpage
\thispagestyle{empty}  

~ \\


\newpage

\section*{Pre\'ambulo}

El presente documento sigue las indicaciones del BOJA Número 88 de 12 de mayo de 2025. Se establece la presentación de un proyecto con los planteamientos docentes e investigadores del candidato. Se debe respetar la extensión máxima de 50 páginas en A4 con letra de 12 puntos de cuerpo, con espaciado interlineal de 1.5 y márgenes de 2.5 cm.

Atendiendo a estas indicaciones el documento se divide en los siguientes apartados:

\begin{itemize}
\item Introducción
\item Planteamientos docentes
\item Planteamientos investigadores
\end{itemize}

\vfill


Mario G\'omez Ramos

En Sevilla, a \today.

~ \\

~ \\

~ \\


\newpage

~ \\
%
%
\newpage

\tableofcontents
\newpage

~ \\




\section{Introducci\'on}
\label{intro}

El presente documento desarrolla los planteamientos docentes e investigadores relativos al desempeño como Profesor Permanente Laboral dentro del sistema universitario. Naturalmente, es necesario que estos planteamientos se ajusten a la concepción actual de la Universidad y busquen la consecución de sus objetivos y funciones, por lo que es conveniente analizar cuáles son estos objetivos y funciones para una universidad moderna.

La universidad de Bolonia, fundada en 1088, se considera la primera universidad del mundo, al ser la primera institución con las siguientes características:

\begin{itemize}
\item Era capaz de expedir titulaciones académicas de alto nivel.
\item Utilizó explícitamente la palabra \textit{universitas}, de \textit{niversitās magistrōrum et scholārium} (comunidad de Profesores y académicos), en su fundación.
\item Era formalmente independiente de la educación eclesiástica.
\item Ofrecía estudios seculares, como gramática, lógica o derecho.
\end{itemize}

Si bien la Universidad de Bolonia, y otras fundadas en el periodo medieval, como las Universidades de París (1150), Oxford (1096) o Salamanca (1218) presentan características asociadas actualmente con las instituciones universitarias, como la libertad de cátedra, su concepción y funciones eran significativamente distintas de las de una universidad actual. Por ejemplo, la investigación, ahora íntimamente ligada al entorno universitario, tuvo que esperar a 1794 para su primera vinculación oficial a la universidad, con la creación de la primera cátedra de investigación científica por la Universidad de Cambridge.

Por lo tanto, para una concepción clara de la misión de la universidad es necesario centrarse en visiones más recientes. Por ejemplo, para Ortega y Gasset, en su trabajo \textit{Misión de la Universidad}\cite{ortega_gasset}, las funciones de la universidad son las siguientes:

\begin{itemize}
\item Ser transmisora de cultura y de la realidad de una época
\item Formar futuros profesionales y prepararlos para esta labor
\item Investigar y formar a los estudiantes en la ciencia
\end{itemize}

Por otro lado, la UNESCO, en su \textit{Declaración Mundial sobre la Educación Superior en el siglo XXI} \cite{unesco}, expande las funciones de la universidad:

\begin{itemize}
\item Función de educar, formar y realizar investigaciones
\begin{itemize}
\item Formar diplomados altamente cualificados y ciudadanos responsables.
\item Constituir un espacio abierto para la formación superior que propicie el
aprendizaje permanente.
\item Promover, generar y difundir conocimientos por medio de la investigación.
\item Contribuir a comprender, interpretar, preservar, reforzar, fomentar y difundir
las culturas nacionales y regionales, internacionales e históricas.
\item Contribuir a proteger y consolidar los valores de la sociedad.
\item Contribuir al desarrollo y la mejora de la educación en todos los niveles.
\end{itemize}
\item Función ética, autonomía, responsabilidad y prospectiva
\begin{itemize}
\item Poder opinar sobre los problemas éticos, culturales y sociales, con total
autonomía y plena responsabilidad.
\item Reforzar sus funciones críticas y progresistas.
\item Utilizar su capacidad intelectual y prestigio moral para defender y difundir
activamente valores universalmente aceptados como la paz, la justicia, la libertad, la igualdad y la solidaridad.
\item Disfrutar plenamente de su libertad académica y autonomía siendo al mismo tiempo
plenamente responsables para con la sociedad.
\item Aportar su contribución a la definición y tratamiento de los problemas que
afectan al bienestar de las comunidades, las naciones y la sociedad mundial.
\end{itemize}
\end{itemize}

Más reciente es el \textit{Comunicado de la Conferencia de Ministros de Educación Europeos} \cite{Com2009}, donde se reconocen como misiones de la universidad:

\begin{itemize}
\item  Preparar a los alumnos
para la vida como ciudadanos activos en una sociedad democrática
\item Preparar a los alumnos para su carrera profesional futura y permitir su desarrollo personal
\item Crear y mantener una
amplia y avanzada base de conocimiento
\item Estimular la investigación y la innovación
\end{itemize}

A partir de todos estos documentos, se hace patente que diversos actores e instituciones presentan distintas visiones sobre la función de la universidad en la sociedad, si bien hay dos pilares que surgen consistentemente, asociados con la generación y transmisión de conocimiento por parte de la comunidad universitaria: investigación y docencia. Son estas dos funciones del personal universitario las que se discutirán en las siguientes secciones.

\section{Planteamientos Docentes \label{sec:docente}}

En esta sección se presentan los planteamientos docentes del candidato. Se presenta el contexto normativo e institucional de las legislaciones relevantes y se presenta la Universidad de Sevilla como institución particular en la que se realizará la labor docente, como contexto sobre el que se introducen los planteamientos, que, aunque se presentan de manera general, en los casos en los que sea necesario se particularizarán a la asignatura de Física Nuclear y de Partículas. 

\subsection{Marco normativo e institucional}

\subsubsection{Espacio Europeo de la Enseñanza Superior}

El Espacio Europeo de la Enseñanza Superior (EEES) surge en un esfuerzo por armonizar la educación superior en distintos países de la Unión Europea y mejorar su competitividad. Siguiendo los principios fundamentales expuestos en la \textit{Magna Charta Universitatum} \cite{magna_charta}, firmada en Bolonia en 1988, los objetivos del EEES son los siguientes, recogidos en la declaración de Bolonia de 1999 \cite{bolonia}:

\begin{itemize}
\item Adopción de un sistema de títulos fácilmente comprenstibles y comparables.
\item Adopción de un sistema basado esencialmente en dos ciclos principales.
\item Puesta a punto de un sistema de créditos, como el sistema ECTS, para promover una mayor movilidad entre los estudiantes.
\item Promoción de la movilidad mediante la eliminación de los obstáculos al ejercicio efectivo del derecho a la libre circulación.
\item Promoción de la cooperación europea en materia de aseguramiento de la calidad.
\item Promoción de la necesaria dimensión europea en la enseñanza superior.
\end{itemize}

Esta declaración de Bolonia fue firmada por los ministros de 29 países y actualmente son 49 los países que forman parte del EEES, con reuniones periódicas, la última de las cuales tuvo lugar en Tirana (Albania) en 2024. Pilares fundamentales del EEES son los créditos ECTS, el sistema de dos ciclos (grado y posgrado) y el cambio en la metodología docente, que se describen a continuación:

\textbf{El sistema de créditos ECTS (Sistema Europeo de Transferencia de Créditos)} 

Para favorecer la movilidad de los estudiantes entre distintos centros del EEES y la convalidación de los estudios realizados en ellos, es necesario definir una unidad de medida común para el trabajo desempeñado por el estudiante en base a la cual se pueda considerar si el estudiante cumple los requisitos necesarios para la concesión del título correspondiente. Dentro del EEES, esta unidad de trabajo se corresponde con el crédito ECTS, de las iniciales en inglés de Sistema Europeo de Transferencia de Créditos. Debido a que las competencias de educación corresponden a los estados, distintos estados dentro del EEES pueden presentar distintos requisitos sobre el trabajo que corresponde a 1 ECTS. Dado que este proyecto se corresponde a una universidad española, se presentan los criterios recogidos por la legislación española, introducido por el Real Decreto 1125/2003 [168] de 5 de septiembre \cite{dec_ects}. De acuerdo a este decreto, un crédito debe corresponder a 25-30 horas del trabajo de un alumno, incluyendo trabajo dentro y fuera del aula, con un número de créditos total para un curso académico de 60 a repartir en unas 36-40 semanas anuales, resultando en una dedicación de unas 37.5 horas, es decir, a tiempo completo.

Dado que el trabajo personal del alumno para cumplir con los objetivos de una asignatura dependerá de las circunstancias y capacidades de dicho alumno, esta relación entre horas de trabajo y número de créditos ECTS debe tomarse como una primera aproximación, siendo parte de la labor docente ajustar los objetivos de las distintas asignaturas para que un alumno medio pueda alcanzar estos objetivos con una dedicación próxima a la teórica. Es destacable que a la hora de asignar créditos de docencia al profesorado no es habitual tomar el trabajo personal del docente en consideración, de manera que créditos que corresponden con cargas de trabajo significativas para el docente, como aquellos asociados a trabajos fin de curso, suelen infrarrepresentar el esfuerzo por parte del docente. En este sentido, sería deseable una mejora en los sistemas de asignación de créditos al profesorado.

\textbf{Ciclos de grado y posgrado}

El sistema del EEES presenta la reestructuación de la educación superior en cursos de grado y posgrado, los primeros habilitando al estudiante a acceder al mercado laboral y a los estudios de posgrado, y los segundos correspondiendo a una mayor especialización. Dentro del sistema español, el estudio de posgrado se corresponde a los estudios de máster. El acceso a los estudios de doctorado requiere la compleción de 300 ECTS entre estudios de grado y máster, siendo indispensable el curso de un máster previo al doctorado. Estos estudios están actualmente regulados por el Real Decreto 822/2021, de 28 de septiembre \cite{dec_master}.

En lo que corresponde al grado, son estudios que prestan \textit{enseñanzas básicas y de formación general, en una o varias disciplinas, orientadas
también a la preparación para el ejercicio de actividades de carácter profesional} y la gran mayoría de los estudios corresponden a 240 créditos ECTS (4 años), aunque pueden ser de 180 ECTS si las universidades arbitran mecanismos para complementar el número de créditos hasta 240 a través de titulaciones de máster, y pueden ser superiores en casos especiales, i.e. el grado en Medicina, cuyos créditos totales son 360.

Los estudios de máster buscan \textit{la adquisición por el estudiante de una formación avanzada, de carácter
especializado o multidisciplinar, orientada a la especialización académica o profesional, o
bien a promover la iniciación en tareas investigadoras} y pueden corresponderse a 60, 90 ó 120 ECTS.

Dentro de la legislación vigente, tanto los estudios de grado como de máster deben incluir la elaboración de al menos un trabajo obligatorio dentro de la titulación. Para el caso del grado se establece un trabajo fin de grado (TFG) correspondiente a un mínimo de
6 créditos y un máximo del 12,5 por ciento del total de los créditos del título. En el caso
del máster se define un trabajo fin de Máster (TFM) con una duración de entre 6
y 30 créditos.

Es notable que, pese al objetivo integrador del sistema EEES, gran parte de los países pertenecientes favorecen estudios de grado de 180 ECTS con estudios de posgrado de 120 ECTS para alcanzar los 300 necesarios para un doctorado, lo cual dificulta la armonización y convalidación de estudios entre sistemas como el español, donde la distribución es de 240/60.

\textbf{Cambio en la metodología docente}

Como atestigua la introducción del ECTS, en el EEES los objetivos del sistema educativo se estructuran en base al trabajo y el desarrollo del alumno y sus capacidades. Dentro de la idea subyacente a la declaración de Bolonia de una educación continua para los ciudadanos, los contenidos, que pueden evolucionar en el tiempo, toman un segundo lugar frente a las competencias desarrolladas por el alumno. Estas competencias se clasifican en:

\begin{itemize}
\item Competencias genéricas o transversales: aquellas globales que se desarrollan durante la formación universitaria, i.e., resolución autónoma de problemas, trabajo en equipo, comunicación oral.
\item Competecias específicas: aquellas vinculadas a áreas y capacidades
concretas de cada titulación,
\end{itemize} 

y los planes de estudio de titulaciones y asignaturas, además de la evaluación de la consecución de sus objetivos, debe realizarse con el desarrollo de estas competencias por parte del alumno en mente.

\subsection{Las Universidades en España y Andalucía}

En el artículo 27 de la Constitución Española se estipula que:

\begin{enumerate}
\item[1.] Todos tienen el derecho a la educación. Se reconoce la libertad de enseñanza.
\item[2.] La educación tendrá por objeto el pleno desarrollo de la personalidad humana en el respeto a los principios democráticos de convivencia y a los derechos y libertades fundamentales.
\item[10.] Se reconoce la autonomía de las Universidades, en los términos que la ley establezca.
\end{enumerate}

Aun con estas directrices, la competencia en materia de Educación no se indica como exclusiva del Estado, por lo que de acuerdo con el artículo 147, punto 3 de la Constitución, estas competencias pueden ser recibidas por las Comunidades Autónomas, si así lo recogen sus Estatutos. En el caso de Andalucía, efectivamente, el Estatuto de Autonomía, en su artículo 53 reconoce la competencia exclusiva de la Comunidad Autónoma sobre varios puntos asociados al sistema universitario, como la creación y autorización de universidades, programación y coordinación de sistema universitario o la regulación del acceso a las universidades y del profesorado docente investigador, todo esto sin perjuicio de la autonomía universitaria. Esta implicación tanto del Estado como de las Comunidades Autónomas en el sistema universitario, implica que este está sometido tanto a legislaciones estatales como autonómicas, además de los estatutos propios de las distintas universidades

\subsubsection{Ley Orgánica del Sistema Universitario (LOSU)}

La ley estatal vigente que regula el sistema universitario es la Ley Orgánica del Sistema Universitario (LOSU) \cite{losu}, publicada en el BOE 70, del 23 de Marzo de 2023. Los cambios introducidos en esta ley más relevantes para la la convocatoria de esta plaza son la reducción de la temporalidad en los cuerpos docente e investigador de las universidades, imponiendo un límite del 8\% de profesorado temporal como máximo, afectando principalmente a las figuras de Profesor Asociado y Profesor Ayudante Doctor, y la modificación de las figuras docentes estables, desapareciendo la figura de Profesor Contratado Doctor y la creación de la de Profesor Permanente Laboral, correspondiente a este concurso. También es notable el compromiso por incrementar el presupuesto asociado a la investigación, hasta el 1\% del PIB nacional.

\subsubsection{Ley Andaluza de Universidades (LAU)}

A nivel autonómico andaluz, la normativa vigente que cubre las universidades es la Ley Andaluza de Universidades (LAU) de 2003, desarrollada en el contexto de la Ley Orgánica de Universidades (LOU, actualmente sustituida por la LOSU). La LAU fue modificada en 2011 y cuyo texto refundido fue publicado en 2013 \cite{lau}. Dentro de esta Ley se crea un organismo andaluz para potenciar y evaluar la calidad de la educación superior: la Agencia Andaluza de Evaluación de la calidad y la acreditación universitaria (AGAE). Este cuerpo fue reconvertido a la Dirección de Evaluación y Acreditación (DEVA), con registro en el Registro Europeo de Agencias de Calidad (EQAR) en 2014, y una vez más a la actual Agencia para la Calidad Científica y Universitaria de Andalucía (ACCUA), con registro en EQAR en 2024. Entre los objetivos de ACCUA se encuentran evaluar y acreditar a las Universidades, al profesorado y a las actividades de formación e investigación, y, junto con la Agencia Nacional de Evaluación de la Calidad y Acreditación (ANECA), puede emitir acreditaciones para la figura de Profesor Permanente Laboral.

Debido al nuevo entorno legal introducido por la LOSU, la junta de Andalucía esta tramitando una nueva ley de universidades, la Ley Universitaria para Andalucía (LUPA), cuyo anteproyecto fue publicado el 3 de octubre de 2024 \cite{lupa}, entre cuyas reformas destacables se halla la restitución de la figura Profesor Contratado Doctor.

\subsection{La Universidad de Sevilla}

La Universidad de Sevilla, fundada en 1505, es la tercera de España en número de alumnos, por detrás de la UNED y la Universidad Complutense de Madrid \cite{cifras}, con $\sim$71000 alumnos matriculados, $\sim$4500 profesores y una oferta de 89 grados, 117 máster y 32 programas de doctorado en el curso 2023-2024 y está compuesta por 32 centros y 134 departamentos \cite{us}. La Universidad de Sevilla presenta buenas posiciones en rankings internacionales como el Academic Ranking of World Universities o ránking de Shanghai (485), el QS World University Ranking (462) o el NTU Ranking (447) \cite{us}.

Además de por las normativas estatales y autonómicas, la Universidad de Sevilla se rige por estatutos propios, cuya última versión ha sido publicada en el BOE el 20 de mayo de 2025 \cite{estatutos}. En el artículo 14 de dichos estatutos se establece la estructura académica de la Universidad de Sevilla, integrada por:

\begin{enumerate}[label=\alph*)]
\item Centros: facultades y escuelas. Asimismo, tendrán la consideración de centros la Escuela Internacional de Doctorado, la Escuela Internacional de Posgrado y el Centro de Formación Permanente.
\item Departamentos.
\item Institutos universitarios de investigación.
\item Centros mixtos e instalaciones científicas.
\item Otros centros y estructuras que se creen para el desarrollo de las funciones de la Universidad.
\end{enumerate}

Dado que la plaza asociada al presente concurso está asociada al Departamento de Física Atómica, Molecular y Nuclear de la Facultad de Física de la Universidad de Sevilla, los siguientes apartados corresponden a dicho centro y departamento.

\subsubsection{Facultad de Física}

En la facultad de Física se imparten las siguientes titulaciones de grado:
\begin{itemize}
\item Grado en Física
\item Grado en Ingeniería de Materiales
\item Doble Grado en Física y Matemáticas (en conjunto con la Facultad de Matemáticas)
\item Doble Grado en Física e Ingeniería de Materiales
\item Doble Grado en Química e Ingeniería de Materiales (en conjunto con la Facultad de Química)
\end{itemize} 

Las titulaciones de máster impartidas son:
\begin{itemize}
\item Máster Universitario en Microelectrónica: Diseño y Aplicaciones de Sistemas Micro/Nanométricos
\item Máster Inter-universitario en Física Nuclear (junto con la Universidad Complutense de Madrid, la Universidad Autónoma de Madrid, la Universidad de Barcelona, la Universidad de Granada, y la Universidad de Salamanca) 
\item Máster Universitario en Tecnologías Físicas para la Medicina y la Biología.
\end{itemize} 

Todos los departamento de la Facultad de Física pertenecen al programa de doctorado de Ciencias y Tecnologías Físicas.

De acuerdo con los Estatutos de la Universidad de Sevilla, es función de la facultad la elaboración de los planes de estudios de las titulaciones impartidas. Además también debe coordinar y supervisar la actividad docente de sus departamentos, que, en el caso de la Facultad de Física son 3:

\begin{itemize}
\item Departamento de Electrónica y Electromagnetismo
\item Departamento de Física Atómica, Molecular y Nuclear
\item Departamento de Física de la Materia Condensada
\end{itemize}

\subsubsection{Departamento de Física Atómica, Molecular y Nuclear}

Al Departamento de Física Atómica, Molecular y Nuclear se encuentran en la actualidad (2025) asociados 76 docentes e investigadores entre personal permanente, personal temporal, contratados predoctorales e investigadores eméritos y honorarios. Este personal se distribuye en tres áreas de conocimiento (si bien la LOSU propone una distribución por ámbitos de conocimiento, esta distribución aún no se ha materializado):

\begin{itemize}
\item Astronomía y Astrofísica
\item Física Atómica, Molecular y Nuclear
\item Física Teórica
\end{itemize}

De estas tres áreas, tanto el perfil docente como el investigador asociados a la plaza correspondiente a este concurso cuadran dentro del ámbito y las asignaturas asignadas al área de Física Atómica, Molecular y Nuclear, por lo tanto, se procede a dar información sobre esta área.

Las asignaturas obligatorias asignadas a nivel de grado al área de Física Atómica, Molecular y Nuclear son las siguientes:

\begin{itemize}
\item Física Cuántica (Grado en Física y Dobles Grados asociados)
\item Física Nuclear y de Partículas (Grado en Física y Dobles Grados asociados)
\item Técnicas Experimentales II (Grado en Física y Dobles Grados asociados)
\item Física I (Grado en Ingeniería de Materiales)
\item Física (Grado en Óptica y Optometría y Dobles Grados asociados)
\end{itemize}

A nivel de máster, el área de conocimiento tiene adscritas asignaturas del Máster Interuniversitario en Física Nuclear, del Máster Universitario Internacional en Física Nuclear de la Escuela Internacional de Posgrado de la Universidad de Sevilla y del Máster Universitario en Tecnologías Físicas para la Medicina y la Biología.

\subsection{Metodología docente}

El cambio de paradigma provocado por la implantación del EEES queda claramente reflejado en el cambio de la focalización en los contenidos del temario por las competencias desarrolladas por el alumnado. El enfoque en competencias no sólo refleja la posibilidad del cambio en los contenidos de una asignatura o plan docente debido a la evolución del campo de conocimiento asociado sino el incrementado protagonismo del alumno en el proceso educativo, ya que son sus competencias las que deben desarrollarse, en lugar de los conocimientos del profesor los que deben ser transferidos. Este cambio es natural en el contexto tecnológico actual, en el que el acceso a la información es fácil, inmediato y casi gratuito y desde la sociedad se reclama que se formen ciudadanos con capacidades críticas y de análisis en un mundo en cambio constante.

A pesar de que este cambio en las bases de la educación pueda parecer exigir con una ruptura con todos los métodos e ideas antiguos, un análisis de los métodos previos al EEES, en particular en carreras de ciencia y tecnología, muestran una similitud significativa entre los métodos antiguos y los nuevos objetivos. Por ejemplo, una competencia transversal requerida en prácticamente todas las titulaciones de ciencias es la resolución de problemas, la cual ha sido la base de las clases (al menos las prácticas) y de la evaluación de asignaturas de carreras de ciencias desde prácticamente su origen. 

Por otro lado, aunque la impartición de contenidos al alumnado, base de los métodos educativos previos al EEES, haya sido denostada en favor de la educación por competencias, debe indicarse que para un adecuado análisis crítico del mundo actual es necesario que el alumno posea un modelo del mundo correcto, es decir, que posea conocimientos correctos de las propiedades naturales y sociales del mundo para que su análisis produzca conclusiones útiles. Sirva como ejemplo una anécdota personal: En un examen de primero de carrera se preguntaba cual sería la temperatura de equilibrio que alcanza un sistema aislado de un bloque de hielo en un vaso de agua a temperatura ambiente, a lo que una de las respuestas respondía que la temperatura de equilibrio era de 30000 K. Sin entrar en las posibles deficiencias matemáticas del alumno, una respuesta tan ridícula parece indicar una incapacidad de análisis de resultados: ¡Por supuesto un vaso con hielo no puede alcanzar 5 veces la temperatura de la superficie solar! Ahora bien, este análisis sólo puede tener lugar si el alumno sabe que una variación de un grado Kelvin es equivalente a la de un grado centígrado, que la diferencia de la temperatura en Kelvin y en grados centígrados es de unos 300 grados y no 300000, que la temperatura ambiente es de unos 30 grados centígrados, que la temperatura solar es de unos 6000 K y que un sistema aislado térmicamente no puede alcanzar una temperatura superior a la máxima temperatura de sus partes únicamente por intercambio térmico. Sin estos conocimientos, el alumno no puede aplicar sus competencias de análisis crítico de resultados, porque no tiene un modelo de mundo razonable con el que comparar. Aun más, no es suficiente que el estudiante tenga acceso a estos conocimientos (en el caso del examen, se proporcionaba graciosamente la relación entre grado Kelvin y grado centígrado y el valor de la temperatura ambiente) sino que el alumno debe tener estos conocimientos interiorizados, ya que un alumno que adolece de un conocimiento, no sabe que adolece de él (esta es una causa del famoso efecto Dunning-Kruger). De hecho existen trabajos que pretenden destacar la complementariedad de la educación por contenidos y competencias \cite{Angulo11}.

La conclusión de estos párrafos es que no es necesario romper frontalmente con los métodos tradicionales de enseñanza en las universidades, sino adaptarlos al nuevo paradigma educativo, dando mayor importancia a la participación y autonomía del alumnado, asegurando que en el proceso educativo desarrolle las capacidades y competencias buscadas.

A continuación se exponen los recursos docentes que el candidato plantea para el proceso educativo, centrándose en la asignatura de Física Nuclear y de Partículas como caso de ejemplo, si bien estos recursos pueden aplicarse a otras asignaturas similares. Es importante reconocer que la eficacia y factibilidad en la aplicación de estos recursos dependerá de factores como el número de horas de clase disponibles, el número de alumnos o el carácter práctico o teórico de la asignatura.

\textbf{Clase magistral} 

En esta clase el profesor expone conceptos y contenidos ante todos los estudiantes de un curso. Este es el método más típico de la educación tradicional y tiene las ventajas de que es efectivo para impartir clase a un número elevado de alumnos, además de permitir al profesor controlar el tiempo dedicado al temario de la asignatura, por lo que es particularmente conveniente para asignaturas con amplio temario y poco tiempo de clase. Tiene la significativa desventaja de reducir al alumno a un puesto de receptor pasivo, impidiendo el desarrollo de sus competencias, más allá de la capacidad de síntesis de los contenidos impartidos por el profesor en la toma de apuntes. Existen métodos para incrementar la participación del alumno en la clase magistral, como facilitarle previamente el material que se expondrá en la lección y realizar preguntas al alumnado para invitar al pensamiento crítico y asegurar el seguimiento de la lección, además de reservar tiempo en la clase para responder a las preguntas que surjan del alumnado. Dado que estas preguntas suelen ser respondidas por los alumnos con mayor capacidad de seguir la lección, puede resultar conveniente realizar preguntas dirigidas a los alumnos, si bien esto puede resultar invasivo para los alumnos más tímidos. Otro método de incentivar la respuesta del alumnado es premiar la participación con puntos extra en la evaluación. Para asegurar el seguimiento de la clase, también es conveniente realizar una introducción con los conocimientos previos requeridos para la lección y finalizar con un resumen del contenido cubierto en la misma.

\textbf{Clase práctica} 

En la clase práctica el profesor plantea problemas y casos prácticos que pueden ser respondidos y analizados aplicando los contenidos de la asignatura. En el caso de la asignatura de Física Nuclear y de Partículas, las clases prácticas se centran en la resolución de problemas y son esenciales para la asignatura ya que permiten desarrollar la capacidad de análisis, pensamiento crítico y resolución de problemas de los alumnos. Es necesario en estas clases centrarse no sólo en la resolución del problema tratado, sino enfatizar en la estrategia a seguir para solventar problemas similares y los métodos para reconocer el tipo de problema que se pretende resolver para una adecuada aplicación de esta estrategia. Además, es fundamental que los alumnos afronten los problemas de manera personal, para poder ejercitar estas capacidades. Para ello, si el número de alumnos es reducido y el tiempo suficiente, se puede proponer que sean los propios alumnos los que presenten la resolución de los problemas que se les habrán facilitado durante la clase. En caso de que las condiciones lectivas no lo permitan, será el profesor el que resuelva el problema, pero siempre asegurando la participación del alumnado, facilitando los problemas antes de la clase para permitir que los alumnos intenten su resolución sin la guía del profesor e incluyendo problemas que no serán resueltos en clase para que los alumnos practiquen su resolución de problemas. 

\textbf{Clases de resolución de dudas} 

Si la distribución de clases lo permite, se pueden reservar horas lectivas exclusivamente para que los alumnos resuelvan las dudas que les han surgido durante el estudio de la asignatura. Estas clases son más efectivas al final del calendario lectivo o de bloques temáticos bien definidos, ya que permiten abordar dudas de todo el temario o dudas globales que surgen desde una visión más general del contenido de la asignatura. Dado que el contenido de las clases depende de las dudas e intereses de los alumnos, que puede ser insuficiente, es conveniente que el profesor acuda a estas clases con puntos preparados, típicamente, cuestiones que, en su experiencia, resultan más complicadas o confusas para el alumnado. Estas dudas pueden extenderse más allá del temario cubierto por la asignatura, y el profesor debe incentivar la curiosidad e interés del alumno con dudas de este tipo, teniendo siempre en mente el interés del resto de alumno y si es más conveniente reservar el tiempo para puntos del temario que requieran clarificación. Si los alumnos son respetuosos e implicados, se puede optar por abrir un foro de debate, donde los alumnos respondan las respuestas de los compañeros, quedando el profesor como mediador de la discusión.

\textbf{Tutorías} 

Desde el Real Decreto 898/1985 \cite{tutorias}, el personal docente universitario tiene la obligación de estar disponible para atender a los estudiantes en horarios de tutoría. La tutoría se corresponde con una atención personalizada al alumno fuera del horario lectivo de la asignatura. Dado que este recurso depende de la solicitud del alumno y que la mayoría de los alumnos no suelen hacer uso de él, es importante que el profesor remarque la disponibilidad y utilidad de las tutorías, que pueden tomar distintas formas. 

\begin{itemize}
\item La \textit{tutoría académica} es la más común, en la que un alumno se reúne con el profesor para discutir dudas y cuestiones que le hayan surgido durante la clase y aclarar dudas. Es destacable que en los trabajos fin de estudios, esta es la forma que toma la práctica totalidad de las interacciones entre alumno y tutor. Esta tutoría permite personalizar la educación universitaria y adaptarla al alumno.

\item La \textit{tutoría docente} tiene lugar entre el profesor y un número reducido de alumnos, típicamente para el tratamiento de temas y dudas que han suscitado dificultad e interés a todos los alumnos que acuden a la tutoría. Aunque consideraciones de tiempo y espacio en el despacho pueden hacer que estas tutorías resulten poco prácticas, la posibilidad de realizarlas virtualmente con los recursos digitales de la universidad hace que sean más factibles para un número más elevado de alumnos o cuando la reunión física resulta imposible.
\end{itemize}

\textbf{Recursos digitales}

Para el apoyo a la labor docente, las universidades, y en particular la Universidad de Sevilla, disponen de numerosos recursos digitales para profesorado y alumnado. Es destacable la plataforma de Enseñanza Virtual, basada en el software Blackboard$^\text{\textregistered}$, que incluye numerosos recursos para la labor docente: permite alojar archivos como diapositivas y hojas de ejercicios, que son visibles únicamente por el profesorado y los alumnos de un determinado grupo, emitir anuncios al alumnado, realizar pruebas de evaluación online y obtener estadísticas al respecto, asignar tareas con fecha límite que los alumnos pueden entregar dentro de la plataforma, crear foros de debate y posee un correo electrónico interno.

Respecto a la búsqueda bibliográfica, los alumnos tienen acceso a un amplio catálogo bibliográfico online a través del catálogo FAMA de la Biblioteca de la Universidad de Sevilla, que además de incluir textos digitales, facilita la búsqueda de libros físicos en las distintas bibliotecas de la Universidad. Además, los alumnos, si están conectados y dados de alta con UVUS en la red de la Universidad, tienen acceso a múltiples revistas científicas que no son open-access pero tienen acuerdos con la Universidad de Sevilla.

Además, la Universidad de Sevilla pone a disposición de los alumnos licencias para software de pago, como el paquete Microsoft Office$^\text{\textregistered}$ o programas de cálculo matemático como Matlab$^\text{\textregistered}$, los cuales resultan particularmente útiles para proyectos más complejos como pueden ser los trabajos fin de estudios.

\subsection{Evaluación}

Con la implantación del EEES y la búsqueda de aprendizaje de competencias, es necesario plantear un sistema de evaluación para el alumnado que refleje satisfactoriamente las competencias adquiridas por el alumno. Dentro de esta perspectiva, parece insuficiente limitar la evaluación del alumno a un examen final en el que demostrar los conocimientos adquiridos tras todo el curso, sin ningún seguimiento que permita al profesorado y al alumnado conocer el grado de consecución de los objetivos de la asignatura hasta el final de la misma, imposibilitando la adecuación de la metodología de enseñanza y estudio durante el curso. Así surge de manera natural el concepto de evaluación continua, en el que un cierto número de evaluaciones durante el curso permiten estudiar la evolución de las competencias del alumnado y actuar en consecuencia.

Dentro del Estatuto de la Universidad de Sevilla se estipula que la evaluación de las asignaturas debe incluir tanto actividades de evaluación continuo como la posibilidad de aprobar una asignatura con exámenes finales. Las actividades de evaluación continua pueden tomar la forma de controles, tareas asignadas, pequeños proyectos, evaluación de la participación en clase u otras posibilidades, mientras que los exámenes, para reflejar adecuadamente la formación en competencias de los alumnos deben incluir partes prácticas que prueben las competencias de análisis, resolución de problemas y pensamiento crítico.

En este punto, es importante mencionar el reciente desafío que la Inteligencia Artificial (IA) y los modelos de lenguaje de gran escala (LLM) como ChatGPT o DeepSeek suponen en particular para la evaluación continua de las asignaturas. El vertiginoso avance tecnológico en estas herramientas hacen que los alumnos con acceso a Internet puedan resolver en cuestión de segundos y sin ningún tipo de trabajo personal gran parte de los ejercicios y tareas propuestos por el profesorado. Las competencias del alumno sólo pueden desarrollarse si este aborda los ejercicios personalmente, desarrollando él mismo las estrategias y métodos requeridos, por lo que un ejercicio resuelto por el alumno con IA no refleja un aprendizaje por parte del alumno. Esto, unido a la posibilidad de ``alucinaciones'' de los LLM (aparición de información errónea recalcitrante que permea las respuestas del modelo) y recientes estudios que apuntan a un efecto deletéreo del uso de los LLM en los procesos de aprendizaje \cite{ZhaiAI,JUAI}, requiere excluir de los procesos de evaluación aquellos trabajos producidos por los LLM. Una analogía útil es imaginar los LLM como un hermano egresado que vive en casa del alumno. Mientras que es aceptable que el alumno se apoye en su hermano para estudiar y resolver dudas, si el alumno permite que su hermano resuelva los ejercicios, difícilmente va a desarrollar el aprendizaje buscado, y la evaluación de ejercicios resueltos por el hermano no puede reflejar el desarrollo de las competencias del alumno.

Para solventar el problema de los LLM existen herramientas que buscan reconocer aquellos resultados producidos por IA. El candidato es escéptico ante su uso, ya que, aparte de situar en una confrontación a profesor y alumno basada en la desconocida fiabilidad de un algoritmo externo, coloca al sistema docente en una carrera armamentística entre los modelos que generan las respuestas y aquellos que buscan reconocerlas. Dada la naturaleza evolutiva de una carrera armamentística, la eficacia de las herramientas de reconocimiento nunca será estable ni segura. Tampoco parece realista confiar en la moral de los alumnos y en su deseo de aprender para que eviten el uso de los LLM, dados los incentivos personales, sociales e incluso económicos que tienen para aprobar una asignatura sin cumplir sus objetivos didácticos.

Desde el punto de vista del candidato, el mejor modo de abordar el problema de los LLM es favorecer la evaluación o bien en ambientes controlados, donde el alumno no pueda tener acceso a los LLM, como sería un examen o un control, o bien en base a un proyecto cuya longitud y complejidad haga impráctica su resolución únicamente en base a LLM, como son los trabajos fin de estudios. Trabajos sin supervisión del alumno, como la resolución de ejercicios en casa, se verían desfavorecidos, a menos que fuera esencial una defensa oral de los resultados por parte del alumno, donde demuestre que ha entendido los pasos requeridos para la obtención de los resultados.

Teniendo esto en cuenta, para la evaluación continua de la asignatura (cuatrimestral) de Física Nuclear y de Partículas se proponen dos controles, a mediados y final del cuatrimestre, correspondientes a los bloques de Física Nuclear y Física de Partículas, como método evaluador principal. Además, también se incluyen controles tipo test de corta duración para proporcionar bonificaciones a los alumnos que consigan superarlos. Estos controles, de rápida resolución y corrección, se realizan a la mitad y el final del temario de cada uno de los bloques, correspondiendo a 4 tests totales, cuyo objetivo principal es fomentar el seguimiento de la asignatura por parte de los alumnos y la evaluación continuada de la consecución de los objetivos de la asignatura.

\section{Planteamientos investigadores}

A continuación se presenta la propuesta de investigación para la plaza en concurso, con una breve discusión previa del contexto institucional y científico asociado a esta propuesta.

\subsection{Contexto internacional, europeo, nacional y autonómico}

La UNESCO reconoce la investigación como función y derecho del personal docente e investigador universitario \cite{unesco}, considerándola como forma principal de promoción del saber, instando a las instituciones a garantizar la formación, recursos y apoyos suficiente para la investigación, encontrando una potenciación mutua entre la misma y la educación superior dentro del seno de las universidades. Además, destaca la necesidad de un equilibrio adecuado entre investigación básica e investigación aplicada.

Dentro de la Unión Europea es el Noveno Programa Marco Horizonte Europa (2021-2027)\cite{horizonte} el que consituye el marco fundamental para la investigación e innovación europea. En él se destacan los siguientes pilares:

\begin{itemize}
\item Ciencia  excelente
\item Desafíos mundiales y competitividad industrial europea
\item Europa innovadora,
\end{itemize}

que buscan garantizar la calidad de la investigación y utilizarla como motor de desarrollo y para la resolución de los problemas sociales actuales.

Dentro de la Física Nuclear es también destacable el Programa de Investigación y Formación de
EURATOM (2021-2025), que busca reducir los riesgos relacionados con la seguridad nuclear.

A nivel del Estado español, la Estrategia Española de Ciencia, Tecnología e Innovación (2021-2027) (EECTI)\cite{eecti}, junto con el Plan Estatal de Investigación
Científica, Técnica y de Innovación (2024-2027) (PEICTI) \cite{peicti}, definen la estrategia general para mejorar la capacidad tecnológica y científica del país, en base al desarrollo de los objetivos europeos. Los objetivos principales de la EECTI son los siguientes:

\begin{itemize}
\item Situar a la ciencia, la tecnología y la innovación como ejes clave en la consecución de los Objetivos de Desarrollo Sostenible de la Agenda 2030.
\item Contribuir a las prioridades políticas de la UE
\item Priorizar y dar respuesta a los desafíos de los sectores estratégicos nacionales
\item Generar conocimiento y liderazgo científico e intensificar la capacidad de comunicación a nuestra sociedad y de influir en el sector público y privado
\item Potenciar la capacidad de España para atraer, recuperar y retener talento
\item Favorecer la transferencia de conocimiento y desarrollar vínculos bidireccionales entre
ciencia y empresas
\item Promover la investigación y la innovación en el tejido empresarial español
\end{itemize}

A nivel autonómico, en Andalucía se encuentra activa la estrategia de I+D+I de Andalucía (2021-2027) (EIDIA 2021-2027) \cite{eidia}, cuyos objetivos estratégicos son:

\begin{itemize}
\item Incrementar el peso de la ciencia y la tecnología en la economía andaluza
\item Aumentar el porcentaje de población dedicada a actividades de I+D
\item Elevar los niveles de transferencia del conocimiento
\end{itemize}

\subsection{Contexto institucional}


\subsubsection{La Universidad de Sevilla}
La Universidad de Sevilla, reconoce en su Estatuto \cite{estatutos} la investigación como un derecho y deber del personal docente, contratado e investigador y confía su organización a la propia Universidad, a los departamentos y a los institutos universitarios de investigación. También indica la función de apoyo de la Oficina de Proyectos de Investigación para la participación de los investigadores en proyectos de investigación, especialmente en el ámbito internacional, presentando un compromiso global de la Universidad con el desarrollo y fomento de la investigación, la innovación y la transferencia de conocimiento.

\subsubsection{Departamento de Física Atómica, Molecular y Nuclear}

El Departamento de Física Atómica, Molecular y Nuclear (FAMN), donde se realizará la labor investigadora asociada a la presente plaza agrupa 7 grupos de investigación:

\begin{itemize}
\item FQM177: Dinámica Estocástica Clásica y Cuántica Aplicada
\item FQM392: Física Interdisciplinar y de no Equilibrio
\item RNM138: Fisica Nuclear Aplicada
\item FQM160: Fisica Nuclear Basica
\item FQM196: Nanotecnología en Superficies y Plasma
\item FQM112: Mecánica Estadística
\item FQM402: Ciencias y Tecnologías del Plasma y el Espacio
\end{itemize}

La presente propuesta investigadora se engloba dentro del grupo FQM160: Fisica Nuclear Basica, en el que el estudio de reacciones nucleares consiste una de las líneas de investigación principales.

\subsection{Propuesta de investigación para el perfil}

El perfil investigador de la plaza es ``Estudio teórico de
reacciones nucleares directas a energías bajas e intermedias''. El candidato presenta a continuación una justificación de su adecuación a dicho perfil así como las distintas líneas de investigación relacionadas con el perfil que pretende desarrollar.

\textbf{Líneas de investigación}

A continuación, se describe la propuesta investigadora del candidato. La propuesta
está dividida en cuatro líneas de investigación (L1-L4) alineadas con los antecedentes previamente descritos, la experiencia del candidato y el perfil investigador de la
plaza.

\underline{L1: Reacciones de arranque de una partícula: reacciones $(p,pN)$, \textit{knockout} y transferencia}

Dado el poder espectroscópico de las reacciones de arranque de una partícula, es esperable que dichas reacciones sigan siendo un importante foco de interés experimental en el futuro, ya que permiten el estudio de problemas de importante interés para la comunidad de física nuclear, como la evolución de las capas nucleares en núcleos lejos del valle de betaestabilidad y las correlaciones entre nucleones de corto y largo alcance \cite{nupecc}, a través de la reducción o \textit{quenching} de los factores espectroscópicos. Para poder utilizar los valores experimentales de los factores espectroscópicos para inferir información sobre las correlaciones entre nucleones es necesario primero entender bien la influencia del formalismo de reacción, como se ha indicado en los antecedentes. Por lo tanto, el candidato propone la continuación de su exploración de los efectos de destrucción del core en reacciones de \textit{knockout} de una partícula \cite{quenching}, extendiendo su estudio a más núcleos para los que existen medidas experimentales \cite{Tos21} e intentando cuantificar las incertidumbres provocadas por el uso de potenciales ópticos mediante el uso de varias prescripciones realistas. Dado que el método utilizado hasta la fecha es un método semiclásico \cite{quenching}, el desarrollo de expresiones completamente cuánticas sería valioso para evaluar los sesgos provocados por la aproximación semiclásica. Estos proyectos tendrán el apoyo de los co-autores del departamento de FAMN de los estudios previos: J.~Gómez Camacho y A.M.~Moro.

Gracias a los avances en blancos de hidrógeno, \cite{minos,onokoro}, y en particular con el desarrollo del proyecto ONOKORO, es esperable que en el futuro cercano se realicen numerosas reacciones en cinemática inversa con blancos de protones, de las cuales las reacciones de arranque de nucleones $(p,pN)$ podrán ser fácilmente medidas gracias a su sección eficaz relativamente elevada y a su clara señal experimental (nucleón blanco y nucleón arrancado son emitidos con un ángulo relativo próximo a los 90$^\circ$). La experiencia del candidato en reacciones $(p,pN)$, foco de su tesis doctoral, será muy relevante en el estudio de estas reacciones. Dado que los núcleos de interés corresponden a un rango de masas medio $A\sim 50-200$, los cálculos de la estructura de dichos núcleos puede resultar dificultoso, para lo cual el candidato buscará el apoyo de T.R.~Rodríguez Frutos, miembro del departamento de FAMN y experto en estructura nuclear.

Para núcleos más pesados es necesario que los cálculos se extiendan a valores del momento angular total mayores $J\sim 100$, para los cuales los cálculos de Transferencia al Continuo empleados se vuelven más inestables debido al incremento de la barrera centrífuga. Por ello, el candidato explorará métodos para reducir esta inestabilidad, tales como la implementación de un cálculo eikonal para valores grandes de $J$ o el desarrollo de un método de extrapolación de las matrices $S$ usando fórmulas asintóticas con una posible combinación con técnicas de IA. El éxito de estos métodos tendría una extensión natural a otros tipos de reacciones, como transferencia o breakup coulombiano, que también sufren de estas inestabilidades.

\underline{L2: Reacciones de arranque de dos partículas: reacciones $(p,3p)$}

Como se ha indicado en los antecedentes y en la línea de investigación anterior, el desarrollo de blancos de hidrógeno de alta resolución ha incrementado el interés por las reacciones con protones, lo que, unido al interés de la comunidad por las correlaciones entre nucleones, hace que las reacciones de arranque de dos protones $(p,3p)$ sean un elemento importante de futuros programas experimentales \cite{onokoro}.

Dado que el candidato desarrolló durante su estancia postdoctoral en la Technische Universität Darmstadt un formalismo semiclásico para el cálculo de secciones eficaces de estas reacciones, se encuentra en una situación perfecta para dar apoyo teórico a los grupos que van a realizar estos experimentos. De hecho, en la actualidad se encuentra en una colaboración con el grupo experimental de la TU Darmstadt con quien realizó la estancia (C.~Xanthopoulos, M.~Enciu y A.~Obertelli) para el análisis de varios experimentos $(p,3p)$ en núcleos en el rango de masas del $^{52}$Ca. Además, ha colaborado en propuestas para los experimentos XXXXX y XXXX con H.~Liu y Y.~Sun, de la Universidad de XXXX.

Aparte de estas colaboraciones, el candidato pretende extender el estudio de las reacciones $(p,3p)$, para las cuales ahora sólo puede calcular secciones eficaces totales para cada estado del núcleo residual. El candidato pretende desarrollar un formalismo semiclásico para el cálculo de distribuciones de momentos del núcleo residual, con aplicaciones espectroscópicas, además de desarrollar un modelo puramente cuántico para la descripción de estas reacciones, cuya aproximación semiclásica puede introducir incertidumbres en los resultados. Como en el caso de las reacciones de arranque de un nucleón, el avance a núcleos más pesados introduce una mayor complejidad en el cálculo de estructura nuclear necesario para la descripción de estas reacciones, por lo que el candidato buscará una vez más el apoyo del profesor T.R.~Rodríguez Frutos del departamento de FAMN.

El candidato también explorará el estudio comparativo de las reacciones de arranque de dos protones $(p,3p)$ con las reacciones de transferencia de dos protones, como $(n,^3$He$)$, dado la amplia historia de las reacciones de transferencia de dos partículas en el estudio de correlaciones entre nucleones. Para ello, el candidato buscará el apoyo del profesor del departamento J.A.~Lay, que posee una amplia experiencia en el estudio de reacciones de transferencia de dos nucleones.

\underline{L3: Desarrollo de la teoría de reacciones a bajas energías}

Las reacciones $(p,pN)$ y $(p,3p)$ se miden a energías intermedias ($\sim$100-500 MeV/A) para maximizar el recorrido libre medio de los núcleones dentro del núcleo y así obtener una reacción ``limpia'' de un solo paso. Sin embargo, el campo de experiencia tradicional del grupo de teoría de reacciones del departamento de FAMN  son reacciones a energías más bajas y de hecho el candidato también posee experiencia en reacciones a energías bajas.

Por ello, el candidato también pretende desarrollar extensiones y mejoras en las teorías utilizadas para la descripción de reacciones a energías más bajas. Una de estas mejoras es la inclusión de la excitación del \textit{core} en reacciones con sistemas de tres cuerpos, punto en el que colabora en la actualidad con el miembro del departamento de FAMN J.~Casal, experto en modelos de tres cuerpos.

Otro punto de desarrollo en la teoría de reacciones a energías bajas es la inclusión de potenciales no locales, punto en el que el candidato ha colaborado con la doctora N.K.~Timofeyuk de la Universidad de Surrey, Reino Unido, previamiente \cite{Tim18,Tim19,Fro23}. La inclusión de la no localidad en teorías estándar de reacción es fundamental para el uso de potenciales \textit{ab initio} obtenidos a partir de primeros principios, que típicamente son no locales y dependientes de la energía. El uso de estos potenciales perimite una mayor consistencia entre las teorías de estructura y reacciones nucleares, por lo que la inclusión de no localidad es un desarrollo importante para la consecución de este objetivo \cite{nupecc}, para lo cual el candidato pretende continuar la colaboración con la doctora N.K.~Timofeyuk.

Particularmente interesante sería la aplicación de potenciales no locales a las reacciones de ruptura no elástica (NEB) \cite{neb} con producción de partículas $\alpha$, que presentan unas sobreestimaciones significativas en la comparación entre teoría y experimento, lo cual requiere la inclusión de no localidad en el formalismo Ichimura-Austern-Vincent (IAV), cuya implementación fue realizada por el doctor J.~Lei y el profesor del departamento de FAMN A.M.~Moro \cite{iav}. El resultado de una colaboración en este punto con ellos podría ser relevante para el estudio de la clusterización $\alpha$ en núcleos ligeros y medianos.

\underline{L4: Apoyo teórico a la comunidad experimental:  desarrollo de herramientas de acceso abierto}

Dada la estrecha relación entre la teoría de reacciones nucleares y los observables experimentales, es fundamental para un físico especializado en teoría de reacciones mantener una estrecha colaboración con los grupos experimentales y servirles de apoyo tanto en la propuesta de medidas experimentales como en el posterior análisis de los experimentos. Como se ha indicado en la sección de antecedentes, el candidato ha participado en numerosas colaboraciones experimentales y actualmente se halla involucrado en colaboraciones con los grupos de R.~Smith, H.~Liu y Y.~Sun para las medidas experimentales de las reacciones XXXXX, XXXXX y XXXX respectivamente.

Dado el compromiso del candidato con la ciencia abierta, el candidato también participa en el desarrollo del código de estructura y reacciones nucleares de código abierto del grupo de teoría de reacciones nucleares de la Universidad de Sevilla, THOx \cite{thox}, en colaboración con los profesores del Departamento de Física Atómica, Molecular y Nuclear A.M.~Moro, J.~Casal y J.A.~Lay. En la versión disponible actualmente, el candidato ha implementado la posibilidad de calcular reacciones con ruptura del proyectil y excitación simultánea de estados colectivos del blanco \cite{target_exc} y el cálculo de secciones eficaces de fotodisociación y captura radiativas para transiciones eléctricas $E\lambda$. También ha paralelizado los pesados cálculos de secciones eficaces de tres cuerpos para reacciones de ruptura (secciones diferenciales angulares en función del ángulo y la energía de emisión de los fragmentos de ruptura).

Futuras mejoras del código THOx en abierto que el candidato pretende desarrollar son el cálculo de canales acoplados con potenciales complejos que den cuenta de la posible ruptura del core debido a su interacción con la partícula de valencia \cite{complex_cdcc} y el cálculo de secciones eficaces en función del ángulo interno protón-neutron para reacciones de ruptura del deuterón, para las cuales ya ha desarrollado implementaciones que deben ser incorporadas a la versión pública del código. Otros puntos de mejora que el candidato ha iniciado pero que aún no cuentan con implementación son el cálculo exacto de estados ligados dentro del modelo de potencial y el cálculo de la probabilidad de excitación magnética B(M$\lambda$) y las secciones eficaces de fotodisociación y captura radiativa asociadas.

Dentro de este punto, también cabe mencionar la participación del candidato en el grupo teórico (Theo4Exp) del proyecto europeo EUROLABS \cite{eurolabs} y su compromiso con el acceso global a cálculos de reacciones nucleares de última generación, mencionado en la sección de antecedentes. Además de su participación en la difusión y soporte de este proyecto, el candidato pretende implementar mejoras en el actual dispositivo online, guíado principalmente por las sugerencias de los usuarios en el reciente taller en el que participó en Julio de 2025 \cite{Workshop_Trento_25}. Estas mejoras incluirían: la posibilidad de seleccionar potenciales globales en los cálculos de reacciones de transferencia, la implementación de más parametrizaciones en la lista de potenciales globales, la introducción de ajuste de mínimos cuadrados para la obtención de potenciales ópticos a partir de datos experimentales o el cálculo de secciones eficaces de captura radiativa. En estos puntos el candidato cuenta con el apoyo de los otros miembros del departamento de FAMN que participan en este proyecto: M.~Rodríguez Gallardo (IP), A.M. Moro, J.A.~Lay y C.T.~Muñoz.

\begin{thebibliography}{99} % '9' is the widest label (e.g., 9 for up to 9 refs, 99 for up to 99 refs)

\bibitem{ortega_gasset}
J. Ortega y Gasset: \textit{Misión de la Universidad.} España: Revista de Occidente, 1$^a$ ed.
Madrid, Sevilla, 1930.

\bibitem{unesco}
UNESCO: \textit{Declaración mundial sobre la educación superior en el siglo XXI},
1998. \url{https://unesdoc.unesco.org/ark:/48223/pf0000113878_spa} (Última visita 01/08/2025)

\bibitem{Com2009} \textit{Comunicado de la Conferencia de Ministros de Educación Europeos}, 2009. \url{https://www.unibasq.eus/wp-content/uploads/2017/10/12.Comunicado_Lovaina_2009.pdf} (Última visita, 01/08/2025)

\bibitem{magna_charta} \textit{Magna Charta Universitatum}, 1988. \url{https://www.magna-charta.org/magna-charta-universitatum/mcu-1988} (Última visita, 01/08/2025)

\bibitem{bolonia} \textit{Declaración conjunta de los ministros europeos de educación reunidos en Bolonia el 19 de junio de 1999}, 1999. \url{https://ehea.info/media.ehea.info/file/Ministerial_conferences/06/0/1999_Bologna_Declaration_Spanish_553060.pdf} (Última visita, 01/08/2025)

\bibitem{dec_ects} \textit{Real Decreto 1125/2003, de 5 de Septiembre}. BOE, 224, 18 de Septiembre de 2003.

\bibitem{dec_master} \textit{Real Decreto 882/2021, de 28 de Septiembre}. BOE, 233, 29 de Septiembre de 2021.

\bibitem{losu} \textit{Ley Orgánica 2/2023, de 22 de marzo, del Sistema Universitario.}. BOE, 70, 23 de Marzo de 2023.

\bibitem{lau} \textit{Decreto Legislativo 1/2013, de 8 de enero, por el que se aprueba el Texto Refundido de la Ley Andaluza de Universidades.} BOJA, 8, 11 de Enero de 2013.

\bibitem{lupa} \textit{Anteproyecto de Ley de Universidades para Andalucía}, \url{https://www.juntadeandalucia.es/servicios/participacion/todos-documentos/detalle/535093.html} (Última visita, 02/08/2025)

\bibitem{cifras} \textit{Las cifras de la educación en España. Curso 2022-2023}, Ministerio de Educación, Formación Profesional y Deportes (2025), \url{https://www.educacionfpydeportes.gob.es/servicios-al-ciudadano/estadisticas/indicadores/cifras-educacion-espana/2022-2023.html} (Última visita, 02/08/2025).

\bibitem{us} \textit{Página oficial de la Universidad de Sevilla}, \url{www.us.es} (Última visita 02/08/2025)

\bibitem{estatutos} \textit{Decreto 98/2025, de 30 de abril, por el que se aprueban los estatutos de la Universidad de Sevilla.} BOE, 121, 20 de Mayo de 2025.

\bibitem{Angulo11} F.A. Angulo y S. Redon, \textit{Competencias y contenidos: cada uno es su sitio en la formación docente}, Estudios pedagógicos (Valdivia) \textbf{37}, 281 (2011), DOI:\url{http://dx.doi.org/10.4067/S0718-07052011000200017 }

\bibitem{tutorias}\textit{Real Decreto 898/1985, de 30 de abril, sobre régimen del profesorado universitario.} BOE, 146, 19 de Junio de 1985.

\bibitem{ZhaiAI}C. Zhai, S. Wibowo and L.D. Li , \textit{The effects of over-reliance on AI dialogue systems on students' cognitive abilities: a systematic review} Smart Learning Environments \textbf{11}, 28 (2024)  DOI:\url{https://doi.org/10.1186/s40561-024-00316-7}

\bibitem{JUAI}Q. Ju, \textit{Experimental Evidence on Negative Impact of Generative AI on Scientific Learning Outcomes} arXiv:2311.05629 (2023) DOI:\url{
https://doi.org/10.48550/arXiv.2311.05629}

\bibitem{horizonte}Comisión Europea, \textit{Horizonte Europa - Invertir para dar forma a nuestro futuro}, \url{https://research-and-innovation.ec.europa.eu/document/9224c3b4-f529-4b48-b21b-879c442002a2_es} (Última visita 03/08/2025)

\bibitem{eecti} \textit{Estrategia Española de Ciencia, Tecnología e Innovación
2021-2027.} \url{https://www.ciencia.gob.es/InfoGeneralPortal/documento/e8183a4d-3164-4f30-ac5f-d75f1ad55059} (Última visita 03/08/2025)

\bibitem{peicti} \textit{Plan Estatal de Investigación
Científica, Técnica y de Innovación (2024-2027)} \url{https://www.ciencia.gob.es/InfoGeneralPortal/documento/80da28ed-be5b-431b-82d5-367d602b713c} (Última visita 03/08/2025)

\bibitem{eidia}\textit{Estrategia de I+D+I de Andalucía (EIDIA), Horizonte 2027}, \url{https://www.juntadeandalucia.es/organismos/universidadinvestigacioneinnovacion/consejeria/transparencia/planificacion-evaluacion-estadistica/planes/detalle/245429.html} (Última visita 03/08/2025)
\end{thebibliography}


\end{document}







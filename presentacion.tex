\documentclass{beamer}

% Tema de beamer
\usetheme{AnnArbor}

% Paquetes
\usepackage[utf8]{inputenc}
\usepackage[spanish]{babel}

% Datos de la presentación
\title{Exposición del CV, proyecto y programa}
\subtitle{Concurso plaza PPL–003–25}
\author{Mario Gómez Ramos}
\institute[]{Universidad de Sevilla \\ Departamento de Física Atómica Molecular y Nuclear \\ Grupo de Física Nuclear Básica}
\date{\today}

\AtBeginSection[]{
  \begin{frame}{Contenido}
    \tableofcontents[currentsection]
  \end{frame}
}

\begin{document}

% Portada
\begin{frame}
    \titlepage
\end{frame}

% Tabla de contenidos
\begin{frame}{Contenido}
    \tableofcontents
\end{frame}

% Sección 1: Curriculum Vitae
\section{Curriculum Vitae}
\begin{frame}{Curriculum Vitae}
    \begin{itemize}
        \item Formación académica
        \item Experiencia profesional
        \item Habilidades y competencias
        \item Publicaciones y logros
    \end{itemize}
\end{frame}

% Sección 2: Proyecto
\section{Proyecto}
\begin{frame}{Proyecto}
    \begin{block}{Título del Proyecto}
        Breve descripción del objetivo principal.
    \end{block}
    \begin{itemize}
        \item Motivación
        \item Metodología
        \item Resultados esperados
        \item Impacto potencial
    \end{itemize}
\end{frame}

% Sección 3: Programa
\section{Programa: Física Nuclear y de Partículas}
\subsection{Temario}
\begin{frame}{Temario}
    \begin{Large}Física Nuclear\end{Large}
    \begin{itemize}
        \item Introducción a la Física Nuclear
\begin{itemize}
        \item L1.1: Introducción
\end{itemize}        
        \item Masas nucleares
\begin{itemize}
        \item L2.1: Fenomenología de las masas atómicas
        \item L2.2: Fórmula semiempírica de masas
        \item L2.3: Límites de formación nuclear
\end{itemize} 
        \item Estabilidad nuclear
\begin{itemize}
        \item L3.1: Introducción y decaimiento alfa   
        \item L3.2: Decaimiento por fisión y beta (I)
        \item L3.3: Decaimiento beta (II)
\end{itemize}      
    \end{itemize}
\end{frame}

\begin{frame}{Temario}
    \begin{itemize}
        \item Tamaños nucleares
\begin{itemize}
        \item L4.1: Medida del tamaño nuclear
        \item L4.2: Reacciones de dispersión elástica de electrones
        \item L4.3: Densidad de carga nuclear
\end{itemize}        
        \item Modelo de capas
\begin{itemize}
        \item L5.1: Introducción al modelo de capas
        \item L5.2: Modelo de partículas independientes
        \item L5.3: Aplicaciones del modelo de capas
\end{itemize} 
        \item Decaimiento gamma
\begin{itemize}
        \item L6.1: Introducción al decaimiento gamma   
        \item L6.2: Reglas de selección y unidades Weisskopf
\end{itemize} 
        \item El deuterón
\begin{itemize}
        \item L7.1: El deuterón
\end{itemize}      
    \end{itemize}
\end{frame}

\begin{frame}{Temario}
    \begin{Large}Física de Partículas\end{Large}
    \begin{itemize}
        \item Introducción a la física de partículas
\begin{itemize}
        \item L8.1: Introducción
\end{itemize}        
        \item Decaimiento y colisiones de partículas subatómicas
\begin{itemize}
        \item L9.1: Fundamentos del decaimiento de las partículas
        \item L9.2: Decaimiento débil
        \item L9.3: Secciones eficaces e interacciones fundamentales
\end{itemize} 
        \item Propiedades de las partículas subatómicas
\begin{itemize}
        \item L10.1: Teoría de Yukawa y clasificación
        \item L10.2: Extrañeza
        \item L10.3: Conservación de números cuánticos y resonancias
        \item L10.4: Isospín
\end{itemize}      
    \end{itemize}
\end{frame}

\begin{frame}{Temario}
    \begin{itemize}
        \item Simetrías discretas
\begin{itemize}
        \item L11.1: Introducción a las simetrías discretas
        \item L11.2: Simetrías P y C
\end{itemize}        
        \item Un paradigma de transición
\begin{itemize}
        \item L12.1: Introducción (somera) a la teoría cuántica de campos
        \item L12.2: Lagrangianos de interacción
        \item L12.3-4: Diagramas de Feynman 
\end{itemize} 
        \item Modelo de quarks
\begin{itemize}
        \item L13.1: Modelo de quarks dentro del grupo SU(3)
        \item L13.2: Descripción de los hadrones en la teoría de quarks
        \item L13.3: Propiedades de las partículas en el modelo de quarks y quarks pesados
        \item L13.4: Diagramas de Feynman en el modelo de quarks
\end{itemize} 
        \item Modelo estándar
\begin{itemize}
        \item L14.1: Introducción al modelo estándar
\end{itemize}      
    \end{itemize}
\end{frame}
\subsection{Bibliografía}
\begin{frame}{Bibliografía}
    \begin{itemize}
        \item Heyde, Kris L. G, \textit{Basic ideas and concepts in nuclear physics: an introductory approach}, Bristol, [England] ; Philadelphia : Institute of Physics Pub., 3rd ed., 2004, ISBN: 9780750309806
        \item Krane, Kenneth S., \textit{Introductory Nuclear Physics}, Ed. John Wiley and Sons, 2nd
ed., 1988, ISBN: 0-471-80553-X
\item David J. Griffiths, \textit{Introduction to elementary particles}, Ed. Wiley-Vch, 2008, ISBN:
978-84-344-0491-5
\item J. Gómez Camacho, \textit{Física de Partículas en 3 créditos}, Universidad de Sevilla. \url{https://idus.us.es/server/api/core/bitstreams/80552150-5cee-40d0-b973-59b3f7cd04a8/content}
    \end{itemize}
\end{frame}

% Diapositiva final
\begin{frame}{Gracias}
    \centering
    \Huge ¡Gracias por su atención!
\end{frame}

\end{document}
